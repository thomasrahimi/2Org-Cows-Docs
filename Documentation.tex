%!TEX TS-program = xelatex
%!TEX encoding = UTF-8 Unicode

\documentclass[a4paper,11pt,oneside]{scrbook} % KOMA script class, which creates an output suitable to the desired book
\usepackage[left=30mm,right=25mm,top=25mm,bottom=25mm]{geometry} %Anpassung der Seitengröße
\usepackage{textcomp} %für die € Zeichen
\usepackage{graphicx}
\usepackage[onehalfspacing]{setspace}
\usepackage{tabularx}
\usepackage{wrapfig}
\usepackage{xltxtra}%wichtig für xelatex
\usepackage{xunicode, xcolor}%wichtig für xelatex
\usepackage{lmodern}%nötig für Durchsuchbarkeit des pdf-Dokuments
\usepackage[margin=10pt,font=small,labelfont=bf]{caption}%Definition der Beschriftungen von Bildern, etc.
\usepackage{setspace}%Absätze mit definiertem Zeilenabstand erstellen
\usepackage{csquotes}
\usepackage{amsmath}%wichtig für Formeln
\usepackage{pdfpages}%wichtig bei Einbindung kompletter pdf-Dateien
\usepackage{listings}
\usepackage{hyperref} %Verlinkung des Inhaltverzeichnisses
\usepackage[automark,headsepline,plainheadsepline]{scrlayer-scrpage} % important to set the page style in the KOMA script book class

\captionsetup[figure]{labelfont=bf,textfont=it,singlelinecheck=false,justification=RaggedRight}
\captionsetup[table]{labelfont=bf,singlelinecheck=false,justification=RaggedRight}

\setmainfont[Mapping=tex-text]{Noto Sans}
\setsansfont[Mapping=tex-text]{Noto Sans}
\onehalfspacing
\sloppy %automatischer Zeilenumbruch

\pagestyle{scrheadings} % creates header compatible with KOMA-script classes, as the file uses KOMA script in its scrbook class
\clearpairofpagestyles % removes footer with page number
\ihead*{\rightmark} % sets title to left side of page header
\ohead*{\thepage} % sets pagenumber to right side of page header
\addtokomafont{pagehead}{\normalfont} % sets font for page header
\renewcommand\chaptermarkformat{\ifnumbered{chapter}{\chapapp\ \thechapter. \ }{}} % sets format for chapter in header

\renewcommand{\labelnamepunct}{\addcolon\space} % replaces the dot in the bibliography with a colon
\renewcommand{\chaptermark}[1]{\markboth{}{#1}} % removes chapter number from header
\renewcommand{\chapterheadstartvskip}{} % Makes chapter titles start right at the beginning of the page

\hypersetup{ %pdf settings
    pdftitle={2Org-Cows Software Documentation},
    pdfauthor={Thomas Rahimi, Boris Kulig},
    pdfsubject={2Org-Cows Software},
    pdfkeywords={Software, 2Org-Cows, Organic Cowbreeding, Database, Big Data},
    bookmarksopen=true,
}

\lstset{ %code display settings
  backgroundcolor=\color{white},
  breakatwhitespace=false,
  breaklines=true,
  frame=single,
  numbers=left,
  numbersep=5pt,
  showstringspaces=false,
  keepspaces=false,
  showspaces=false,
  breaklines=true,
  commentstyle=\color{green},
  tabsize=4,
  numberstyle=\tiny\color{gray},
  basicstyle=\fontsize{10}{12}\ttfamily
}

%opening
\title{2Org-Cows Software Documentation}
\author{Thomas Rahimi: \href{mailto:thomas.rahimi@mailbox.org}{thomas.rahimi@mailbox.org},\\ Boris Kulig: \href{mailto:bkulig@uni-kassel.de}{bkulig@uni-kassel.de}}

\date{}




\begin{document}

\maketitle
\begin{center}
 Universität Kassel,\\
 Fachbereich 11,\\ 
 ökologische Agrarwissenschaften Witzenhausen,\\
 FG Agrartechnik
\end{center}

\bigskip

\begin{center}
 \includegraphics[width=5cm,keepaspectratio=true]{./Logo-2Org-Cows.png}
 % Logo-2Org-Cows.png: 128x128 pixel, 72dpi, 4.51x4.51 cm, bb=0 0 128 128
\end{center}
\thispagestyle{empty}
\newpage

%\clearpage
\thispagestyle{empty}
\phantomsection %wichtig für die richtige Verlinkung im Inhaltsverzeichnis, muss vor dem zu verlinkenden Artikel sein.
\addcontentsline{toc}{chapter}{Table of Contents}
\tableofcontents
\phantomsection
\listoffigures
\addcontentsline{toc}{chapter}{List of Figures}
\thispagestyle{empty}
\newpage

% !TEX root = Documentation.tex

\section{Introduction}

The 2Org-Cows project is multinational and multi-partner project of research institutions in organic agriculture in Europe. It aims to improve the breeding of dairy cattle for organic dairy cow keeping
and therefore aims to collect data to measure different parameters of animal welfare, health and keeping.\\
In order to analyze all these data, a huge database has been created, which consists of the following components:
\begin{itemize}
 \item a webbased, graphical user interface for the final user to access and query the database using a standard webbrowser
 \item a variety of interfaces to other programs
 \item software to analyze database stored datasets using big data methods, mainly based on R and on Python3
 \item MySQL databases as the storage backbone of the whole software and project
 \item Ubuntu 16.04 operating system, PHP7 scripting language and Apache2 webserver to provide the user with the data required
\end{itemize}
In the following all parts of the software for the 2Org-Cows project should be documented. Please note that this file is work in progress and will therefore change steadily.

\section{Common information regarding the coding}
The project is generally based in its programming on several concepts, which should be briefly displayed in the following. The programming is therefore differenciated in the parts 
regarded. 
\paragraph{Web Frontend}
The web frontend is largely written in PHP7 and HTML5, following the following paradigms: 
\begin{itemize}
 \item clear structure of code
 \item clear separation of display code and code for the processing of input
 \item easy to write, understand and maintain code
 \item mostly procedural code style, only exception is the database system, which is based on object oriented code style
\end{itemize}


\newpage

% !TEX root = Documentation.tex

\section{The webbased User Interface}
The user interface for the database has been developed as a graphical interface to allow broad acces to the database. In the following, the different pages and their corresponding 
scripts, in the web frontend, are described in detail.


\subsection{login.php}
The login.php site serves as the welcome site of the whole project, including basic information on the function of the sites for user access to the database.\\
For security reasons and to track the visitors of the page, basic information about the visitor is stored in a separate database. This includes IP addresses, user agent of 
the browser and the time of access. To comply with the lawful requirement not to store personal information, such as full IP addresses, the IP address is shortened to three blocks,
which does not allow single user identification for non-logged-in users.\\
The page also starts a short test session, to check, whether cookies are are allowed in the browser. The result is not completely true, as it technically requires a reload of the 
page, which is not forced by the script.
The login works with HTML5 forms, organized in a table to allow easier configuration to various screen sizes and offer the entering of username and password. 
These forms consists of a text input with the input widget for text, to enter the username and a password input with the 
password widget. To prevent CSRF-attacks on the visitors of the page, a random sequence is generated by the server and linked to the page. This sequence is requested in the 
processing of the login during login\_script.php to check for CSRF-attacks and to issue an alert if the sequences do not match. 
Furthermore there is a submit button, which is submits the form data. Submitting this form starts processing of the form data with login\_script.php .

\subsection{login\_script.php}
The login\_script.php script processes the data submitted from the login form. For security reasons, 
authentication makes use of two different databases. The authentication database does not include any other data than the data required to initiate a session. In contrast,
the agri\_star\_001 database includes more details on the specific user, including the insititution and the role of the user.\\
To ensure the integrity of the user request for authentication the form on login.php includes a random sequence, which is both stored, in the session and the POST submission. 
The comparison of the both results should ensure better protection from XSS-attacks.\\
Data processing happen in 4 steps, which come up sequencially.
\begin{enumerate}
 \item At first the script checks, whether a username exists, which matches the username provided in the form. In case the username can not be found in the database ``auth'', the script
 redirects the user to login.php and creates error message, indicating unknown username. Otherwise the script continues to step 2.
 \item In case, a user has tried, a login with non matching credentials, too many times, the user is redirected to the login.php page and told to wait for some time. This settings 
 should slow down brute force attacks on available accounts.
 \item If the username has been found, the script checks, whether the password, provided by the user, matches the password stored in the ``auth`` database. Therefore the PHP function 
 password\_verify() is used. If the password provided does not match to the hash value, stored in the database, the user is redirected to login.php, receiving an error message about the 
 password. Otherwise the script starts to initiate a session.
 \item In the next step the script starts to create a session for the use of the research database. In order to ensure the security of the session, several steps are taken:
 \begin{itemize}
  \item the IP address of the user is stored as a session variable to prevent session hijacking
  \item the user agent of the user's browser is also stored as a session variable to prevent session hijacking
  \item the user's institution is stored to regulate access to the data stored in the research database, this information is encoded in the group id, which is set as a session
  variable, instead of the group name and the department
  \item the username is stored into the session to allow recognition of the user 
  \item the user-id is stored into the session, because it is the primary key to the users in the different databases
  \item the session-id is stored into the session aswell, to allow the recognition of the session in the further session administration system
 \end{itemize}
  Afterwards, the user is redirected to home.php, which acts as the main page. In order to check on the usage of the website, all successful logins are logged into a separate 
  table in the logging database. To minimize the tracking of single users according to their IP address, the IP address is shortened to three blocks, which still allows sufficient 
  data for the analysis of user interaction with the database.
  \item In case no login information has been provided by the user, the user is redirected to login.php .\\
\end{enumerate}
In case the login failed, the attempt is written into the Login\_Failed table in the logdb database, to track bruteforce attacks on user accounts.\\
The database connections are closed at the end of each script to prevent accumulation of unused database connection by various scripts.

\subsection{check-session.php and check-session\_restricted.php}
The both scripts check-session.php and check-session\_restricted.php act as the connecting scripts between the different pages within the project. The basic intention of the scripts is to check, whether a session for the user exists and whether the user, therefore, is allowed to access pages within the project. Furthermore, the check-session\_restricted.php script checks, whether the logged in user has sufficient 
permissions, to access limited pages. 


% !TEX root = Documentation.tex

\subsection{home.php}

The home.php page serves as the central page of the whole project, providing basic information on the use of the frontend and showing all navigation entries to refer to other pages 
within the frontend. To ensure that non permissioned users can not access pages such as \hyperref[create-user.php]{create-user.php}, these entries are just shown to users with roles 
higher than 1 or 2. The same happens for the menu, which is included in all pages, to allow easy navigation. 

\subsection{traffic.php}
The traffic page is the evaluation page to the logdb database, which keeps track of users accessing the login.php page and successful logins. As these data are highly sensitive and 
not important for the daily use of the database, the access to this page is limited to users with the role admin (role code 4). The presented data are split into different tables, 
one for the login.php page, displaying amount of accesses, time and IP address. Another table displays the data for the logged in users, showing IP address, time of login and user 
invoked. And finally there is a table displaying all login attempts, which failed due to incorrect credentials. This table should enable the recognition of brute force attacks on 
given accounts.

% !TEX root = Documentation.tex

\subsection{create-user.php}
The create-user.php site offers the opportunity to create a new user and to set the right for this specific user. User creation follows the idea of dropping rights, 
which only allows a user to create another user with less rights than he has himself. To ensure this, create-user.php is just visible to users with a role, higher than student.
This includes the following roles: admin [role code: 4], professor [role code: 3], scientific coworker [role code: 2]. \\
Furthermore only professors and admins are allowed to create users at a different institution, scientific coworker are only allowed to create users at the 
same institution as they are. In the same way, the erasing of users is limited to professors and admin users.\\
Creating a user leads to modifications in the Dim\_User table and the authentication database. This database updates are proceeded, if the user clicks on 
the ``create user'' button, which calls the ``create\_user\_script.php'' file to process the form data and insert them into the database.\\ 
The creation of new institution is also limited to people logged in as professors and admin. Each department is created separetely, even if an institution owns several departments, 
taking part in the 2Org-Cows project. This improves the adjustability of rights sticked to the specific department in case of splitted user rights on certain datasets.
The data inserted for the creation of an institution are processed by the create\_institution\_script.php file.

\subsection{create\_user\_script.php}
The create\_user\_script.php is the processing file for the entries to create a new user in the web interface for the database. Basically the script just performs the validity checks 
of the form data provided in create-user.php and an \textbf{INSERT} query to the MySQL database.\\
At the beginning the script checks whether the provided check bytes submitted in create-user.php match those from the session. Again this is a CSRF-protection in case the session has
been hijacked. If there are no problems on the CSRF protection, the script checks for the validity of the form provided. This part already takes place in the html, where the tag 
``required'' is set for all fields, but a controll in the script should avoid empty fields in the database.\\
For the validity checks the following steps are taken:
\begin{itemize}
 \item validity check of the e-mail adress provided
 \item check whether the desired username is available
 \item check, whether the password matches matches the control field
 \item check whether the user permissions, entered in the form, are allowed to be performed by the user invoked
 \item check whether the user invoked is allowed to select an institution different from his own
\end{itemize}
If any of these checks fails, the user receives an error notice and is directed to the form.As the group, department and the country entries are stored in a different
way in the database, than in the user session, the database values have to be retrieved from the database. Therefore, the corresponding institution, department and country are 
queried from the Dim\_Group table. As the country is stored as an integer value, the corresponding ID\_Country has to be queried from the Dim\_Country in the next step.\\
The passwords are stored as bcrypt hash values in a separate database, the hashing is performed after the validity checks for the form. The next step is to insert all data into the 
Dim\_User table in the agri\_star\_001 database. The insert is proceeded in an objective style, generating the query first, then assigning the values to the query 
(mysqli::prepare(\$query), mysqli::bind\_param(parameters)) and than executing the query (mysqli::execute). The transaction id of that insert is retrieved, as it is used as ID\_User
to identify the user in the various databases and therefore also in the auth database. In the next step the username, the password hash and the ID\_User are inserted into the 
auth database, using the same process as for the Dim\_User table.\\
The scripts quits with a success message.

\newpage

% !TEX root = Documentation.tex

\section{Interfaces}
The interfaces to the database are largely written in Python3 and implemented to act independent from the user interface. This separation should minimize ressource conflicts between the user 
interface and the database interfaces, if large files are processed. As the files are largely written in Python3, the coding style is oriented on the recommendations from 
\href{https://dev.mysql.com/doc/connector-python/en/}{\textit{MySQL developer guide}}. The connection to the database is established with MySQL Connector, which has to be installed separetely on Ubuntu 
systems (see the \hyperref[Database installation and configuration]{\textit{database section}}). 
As a security measure and to ensure easy enrolling of code in different environments, the provision of database access credentials to the code is done with separate files. The selected option is to 
provide the credentials from a *.json file, which is evaluated with the native json library in Python3. In the next step, the connection to the database is established within the corresponding 
function.\\


\newpage

% !TEX root = Documentation.tex

\section{Database}

The database for the projects are API-compatible, SQL databases, which run on freely available linux systems. Eventually the used and tested MySQL or MariaDB databases could be 
replaced by other systems, such as MSSQL. Nevertheless, a full compatibility of other databases with the source code provided can not be guaranteed. To ensure safe operation of the
server, databases, which are not directly linked to the project data, are separated from the main database. As a result, the login database and the external logging database are 
separated from the main database. Therefore, three different databases exist for the purpose of the project.\\ 
To increase the security of the database and to limit the risk of damage in case of SQL-injections, all databases are accessed by non-privileged users. Furthermore, there is a 
separation in the users, regarding the function of the database invoked, to limit access to sensitive data in the database.

\newpage

\input{jupyter}

\newpage

% !TEX root = Documentation.tex

\section{Operating System}
\subsection{Ubuntu}
In Ubuntu less modifications than in CentOS are necessary to allow a stable and safe operation of the server. As the operating system ships another mandatory access controll system
(MAC), configuration of user rights is far easier and faster proceded. During the installation of the operating system, the LAMP stack and the openssh server have been chosen for
automatic installation. After the installation and the required reboot, updates should be checked and installed.\\

\subsubsection{Securing the server}
To improve the server security, just logged in users can make use of the server. To prohibit external login attempts using ssh, the server is secured in different manners, of which 
one is the setup of groups, allowed to login using ssh. Therefore, a specified group is created, to which all login users are added, requiring the following commands:
\begin{lstlisting}[language=bash]
 sudo addgroup --system [groupname]
 sudo adduser [username] [groupname]
\end{lstlisting}
Than the /etc/ssh/sshd\_config file is modified, including adding and modifying the following lines in the file:
\begin{lstlisting}
 AllowGroups [groupname]
 PermitRootLogin no
\end{lstlisting}
Finally the ssh server has to be restarted.\\
Another security mechanism is the definition of available services in the /etc/hosts.allow and /etc/hosts.deny files on the server. This allows to block access all other webservices, 
except for the required ones from other IP-addresses. This feature has been implemented in parts.\\
Furthermore, the server makes use of a firewall, to prohibit attacks on not required open ports. Therefore, the ufw interface to iptables has been chosen. As the server is not 
required to listen on other ports than the standard ports for the required services, the general rule is to deny all incoming traffic. At the same time, traffic is generally 
allowed with the following commands:
\begin{lstlisting}[language=bash]
 sudo ufw default deny incoming
 sudo ufw default allow outgoing
\end{lstlisting}
In the next step, all required network services, are excluded from the general denial of the services, using the following commands:
\begin{lstlisting}[language=bash]
 sudo ufw allow ssh
 sudo ufw allow http
 sudo ufw allow https
\end{lstlisting}
After the configuration is finished, the firewall is set active with the following command:
\begin{lstlisting}[language=bash]
 sudo ufw enable
\end{lstlisting}


\subsubsection{Database installation and configuration}
As Ubuntu ships MySQL as its default database, MySQL is used in this installation aswell. The database comes with a basic configuration, which should be checked after installation
to match the conditions, under which the database server is operating.\\
If necessary the configuration file under /etc/mysql/conf.d/server.conf has to be altered. It is important to check, if the networking access to the database server is allowed and 
if, which clients are gained access to the databases. For security and compatibility the setting listen: 127.0.0.1 has been chosen here, which allows the network interface of PHP7
to access the database locally, but hinders external clients from initiating a network connection to the database.

\subsubsection{Web server installation and configuration}
Both, apache2 and PHP7 have been installed in the initial installation, together with the database. As the web server should be configured as a virtual host, it is important to 
modify the settings here. There have been two virtual hosts defined, both referring to the same content folder, one host serves as the non-tls demo installation, whereas the other
host is set up with tls support to deliver the results of the requests as encrypted material. The non-tls host is not intended to be used for login, as all user data sent to this host
are in danger of being intercepted. The configuration proceeds in the following steps:
\begin{enumerate}
 \item First the directories for the virtualhost are created, using the following command:
 \begin{lstlisting}[language=bash]
  sudo mkdir -p /var/www/path_to_directory/content
 \end{lstlisting}
 \item Next the permissions on the folder are set to the user invoked or a specific user for the deployment of the web server:
 \begin{lstlisting}[language=bash]
  sudo chown -R $USER:$USER /var/www/path_to_directory/content
 \end{lstlisting}
 \item As the next step, the configuration files for the virtual host are created. To minimize spelling errors, the files are generated from the default configuration, which is 
 copied into a new file:
 \begin{lstlisting}[language=bash]
  sudo cp /etc/apache2/sites-available/000-default.conf /etc/apache2/sites-available/name_virtual_host.conf
 \end{lstlisting}
 \item Now the virtual host has to be configured, with the following settings:
 \begin{lstlisting}[language=bash]
  <VirtualHost *:443>
	SSLEngine On
	SSLCertificateFile /etc/ssl/certs/name_of_cert.crt
	SSLCertificateKeyFile /etc/ssl/private/key.key
	SSLCACertificateFile /etc/ssl/certs/certificate.crt # can be missing, depending on tls configuration of the server
	ServerName www.servername.com
  ServerAlias servername.com
	ServerAdmin admin@e-mail.com
	DirectoryIndex login.php
	ErrorDocument 404 /404 #should avoid 404 page not found errors
	DocumentRoot /var/www/path_to_directory/content
	ErrorLog /var/www/path_to_directory/error.log
	CustomLog /var/www/path_to_directory/access.log combined
	Alias "/admin" /var/www/path_to_directory/content/create-user.php
	Alias "/measurement" /var/www/path_to_directory/content/create-measurement.php
	Alias "/search" /var/www/path_to_directory/content/search.php
	Alias "/results" /var/www/path_to_directory/content/search-results.php
	Alias "/license" /var/www/path_to_directory/content/license.php
	Alias "/upload" /var/www/path_to_directory/content/database-update.php
	Alias "/cow" /var/www/path_to_directory/content/cow.php
	Alias "/logout" /var/www/path_to_directory/content/scripts/logout.php
	Alias "/home" /var/www/path_to_directory/content/home.php
	Alias "/user" /var/www/path_to_directory/content/user-properties.php
	Alias "/404" /var/www/path_to_directory/content/error.php # should avoid 404 page not found errors
  </VirtualHost>
 \end{lstlisting}
  The "Alias" settings maskerade the files, visible to the user with a shorter, easier URL. As the php code mostly refers to the shortened Alias URLs, these entries are essential for the server to work. The \textbf{ServerName} and \textbf{ServerAlias} might be missing, to comply with local DNS settings.
  \item In the next step, all files named in the virtual host configuration, must be created or copied to the server. To create the files, the following command is used:
  \begin{lstlisting}
   sudo touch /var/www/path_to_directory/error.log
   sudo touch /var/www/path_to_directory/access.log
  \end{lstlisting}
  \item To ensure logrotate to compress the log files of the virtualhost, the file /etc/logrotate.d/apache2 has to be altered to the following:
  \begin{lstlisting}[language=bash]
   /var/log/apache2/*.log /var/www/path_to_directory/*.log {
        daily
        missingok
        rotate 14
        compress
        delaycompress
        notifempty
        create 640 root adm
        sharedscripts
        postrotate
                if /etc/init.d/apache2 status > /dev/null ; then \
                    /etc/init.d/apache2 reload > /dev/null; \
                fi;
        endscript
        prerotate
                if [ -d /etc/logrotate.d/httpd-prerotate ]; then \
                        run-parts /etc/logrotate.d/httpd-prerotate; \
                fi; \
        endscript
  }
  \end{lstlisting}
  
  \item In the next step, the encryption certificates for the tls service have to be created. The certificates are self-signed certificates, using SHA256 as hashing algorithm and having a validity of 10 years (3650 days). For this, the following command is used:
  \begin{lstlisting}[language=bash]
   openssl req -sha512 -x509 -nodes -days 3650 -newkey rsa:4096 -keyout /etc/ssl/private/[servername].key -out /etc/ssl/certs/[servername].crt
  \end{lstlisting}
  During the creation process, the information for the certificate have to be provided. Also a Diffie-Hellman file has to be created, to improve the security of the tls connection 
  by using the Diffie-Hellman paradigma for PerfectForwardSecrecy of the sessions:
  \begin{lstlisting}[language=bash]
   sudo openssl dhparam -out /etc/ssl/certs/dhparam.pem 4096
  \end{lstlisting}
  \item In the next step, the tls setting of the whole server are modified to meet most recent requirements on the cryptographic strength of the transaction. Therefore, a 
  designated file under the following location /etc/apache2/conf-available/ssl-params.conf is created. The content of the file is the following:
  \begin{lstlisting}[language=bash]
   #modern security configuration
   SSLProtocol             all -SSLv3 -TLSv1 -TLSv1.1
   SSLCipherSuite          ECDHE-ECDSA-AES256-GCM-SHA384:ECDHE-RSA-AES256-GCM-SHA384:ECDHE-ECDSA-CHACHA20-POLY1305:ECDHE-RSA-CHACHA20-POLY1305:ECDHE-ECDSA-AES128-GCM-SHA256:ECDHE-RSA-AES128-GCM-SHA256:ECDHE-ECDSA-AES256-SHA384:ECDHE-RSA-AES256-SHA384:ECDHE-ECDSA-AES128-SHA256:ECDHE-RSA-AES128-SHA256
   SSLHonorCipherOrder     on
   SSLCompression          off
   SSLSessionTickets       off

   # OCSP Stapling, only in httpd 2.3.3 and later
   SSLUseStapling          on
   SSLStaplingResponderTimeout 5
   SSLStaplingReturnResponderErrors off
   SSLStaplingCache        shmcb:/var/run/ocsp(128000)

   SSLOpenSSLConfCmd DHParameters "/etc/ssl/certs/dhparam.pem"
  \end{lstlisting}
  This file defines the encryption standards, which the server accepts for connections. These standards might exclude older clients, such as Internet Explorer on Windows XP.
  \item To enable the configuration, the following commands have to be executed:
  \begin{itemize}
  \item to enable the ssl module of apache2:
  \begin{lstlisting}[language=bash]
   sudo a2enmod ssl
  \end{lstlisting}
  \item to allow redirection from the non-https site to the https site:
  \begin{lstlisting}[language=bash]
   sudo a2enmod headers
  \end{lstlisting}
  \item to enable the https site:
  \begin{lstlisting}[language=bash]
   sudo a2ensite sitename
  \end{lstlisting}
  \item to enable the ssl-config in apache2:
  \begin{lstlisting}[language=bash]
   sudo a2enconf ssl-params
  \end{lstlisting}
  \item and finally the web server has to be restarted:
  \begin{lstlisting}[language=bash]
   sudo systemctl restart apache2
  \end{lstlisting}
  \end{itemize}
  Other files than those mentioned do not need to be altered to allow steady operation of the server. To allow public reaching of the server, the reuqired ports, which is essential 443, have to be cleared by the firewall.
\end{enumerate}

\subsection{CentOS 7}
In CentOS 7 several modifications were necessary to provide a stable and safe operation of the database and the interfaces. The following describes the installation of the required
packages to a scratch installation. In case a LAMP installation has been chosen during the installation process of the operating system, some of these steps might be superfluous. 
\subsubsection{Database installation and configuration}
First the database had to be installed. 
Therefore \textbf{MariaDB} has been chosen, as it is the default database environment for RHEL-based Linux distributions. For the installation, mariadb-server has to be chosen, all
dependencies are automatically fixed by the yum RPM wrapper. Furthermore PHP7 had to be installed, here it is important
to install the database drivers as well, which are shipped in the package php70w-mysqlnd. Together with PHP7 httpd, as the web server, needs to be installed. For the access of httpd
to the database it is important to execute the following commands:\\
\begin{lstlisting}[language=bash]
 sudo sestatus
 sudo getsebool -a | grep httpd
 sudo setsebool -P httpd_can_network_connect_db 1
\end{lstlisting}

\subsubsection{Web server installation and configuration}
As the LAMP stack is used for the project development, \textbf{apache2} is deployed as the default web server. apache2 is installed as a dependency of PHP7 and is shipped in 
the package httpd. To ensure proper operation, the server has to be configured for shipping of multiple websites beside each other, which is ensured by using virtual hosts. 
Therefore the following steps are necessary: 
\begin{enumerate}
 \item create a folder for the virtual host, using the following command:
 \begin{lstlisting}[language=bash]
  sudo mkdir -P /var/www/path_to_directory/content
 \end{lstlisting}
 \item grant the required permissions on the folder to the user:
 \begin{lstlisting}[language=bash]
  sudo chown -R $user:$user /var/www/path_to_directory/content
  sudo chmod -R 755 /var/www/
 \end{lstlisting}
 \item create configuration files for the virtual host:
 \begin{itemize}
  \item Create the folders for the virtual host configuration:
 \begin{lstlisting}[language=bash]
  sudo mkdir /etc/httpd/sites-available
  sudo mkdir /etc/httpd/sites-enabled
 \end{lstlisting}
 \item Add a line at the end of /etc/httpd/conf/httpd.conf:
 \begin{lstlisting}
 IncludeOptional sites-enabled/*.conf
 \end{lstlisting}
 \end{itemize}
\end{enumerate}


\include{license}

% !TEX root = Documentation.tex

\section{Attachements}


\end{document}
